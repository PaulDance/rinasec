\documentclass[a4paper]{article}

\usepackage[utf8]{inputenc}
\usepackage[T1]{fontenc}
\usepackage[english]{babel}

\author{Paul MABILEAU\\\texttt{paul.mabileau@telecom-sudparis.eu}
        \and Franck STAUFFER\\\texttt{franck.stauffer@telecom-sudparis.eu}}
\title{\textbf{Abstract: Recursive Inter-Network Architecture's Security}}

\begin{document}
\pagenumbering{gobble}
\maketitle

The Recursive InterNetwork Architecture (RINA) is an internet model that has
been presented by John Day in 2008 \textit{Patterns in Network Architecture: A
Return to Fundamentals}.  RINA is presented as an alternative to TCP/IP because
it is claimed to be complex and unsecure.  It is based on Inter-Processus
Communication (IPC) and use a single recursive layer instead of a fixed length
layer stack like TCP/IP\@.  Its main goal is to remove as more variables as
possible from the networking part and to normalize everything.  It uses the
Delta-T communication protocol because it is much simpler than the TCP one but
provides more control than the UDP one.

RINA offers quite strong security mechanisms natively through security policies
that effectively separate protocols from their uses, which in turn allows easy
network behavior control. One of the most important security aspect is
authentication: it is always done with a configurable degree of restriction.
Only then can application processes start to exchange data messages. The
fundamental rewind enables a strict encapsulation of data units, thus avoiding
the need to publish addresses. Also, the use of Delta-T, a soft-state transfer
protocol makes practically infeasible any form of manipulation of connection
states on both ends. Finally, separating protocols and their configuration
policies makes enforcing security much easier because stronger security creates
almost no more additional security mechanisms, thus helping reaching better
overall security sooner.

\end{document}
