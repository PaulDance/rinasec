\documentclass[a4paper]{article}

\usepackage[utf8]{inputenc}
\usepackage[T1]{fontenc}
\usepackage[english]{babel}

\author{Paul MABILEAU\\\texttt{paul.mabileau@telecom-sudparis.eu}
        \and Franck STAUFFER\\\texttt{franck.stauffer@telecom-sudparis.eu}}
\title{\textbf{Abstract: Recursive Inter-Network Architecture's Security}}

\begin{document}
\pagenumbering{gobble}
\maketitle

The Recursive InterNetwork Architecture (RINA) is an internet model that has
been presented by John Day in 2008 \textit{Patterns in Network Architecture: A
Return to Fundamentals}.  RINA is presented as an alternative to TCP/IP because
it is claimed to be complex and unsecure.  It is based on Inter-Processus
Communication (IPC) and use a single recursive layer instead of a fixed length
layer stack like TCP/IP\@.  Its main goal is to remove as more variables as
possible from the networking part and to normalize everything.  It uses the
Delta-T communication protocol because it is much simpler than the TCP one but
provides more control than the UDP one.

\end{document}
