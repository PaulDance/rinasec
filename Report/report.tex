\documentclass[a4paper]{proc}

\usepackage[utf8]{inputenc}
\usepackage[T1]{fontenc}
\usepackage[english]{babel}
\usepackage{url}

\author{Paul MABILEAU\\\texttt{paul.mabileau@telecom-sudparis.eu} \and Franck STAUFFER\\\texttt{franck.stauffer@telecom-sudparis.eu}}
\title{\textbf{Recursive Inter-Network Architecture's Security}}

\begin{document}
\maketitle
\tableofcontents
\newpage
\part*{Introduction}
The Recursive Inter-Network Architecture (RINA) is a network architecture based on Internet Process Communication
(IPC) presented as an alternative to TCP/IP.\@ It was designed a while ago without security in mind. It has been
patched to include security but it creates complexity.\cite{assessing-security}

\part{Comparing RINA against TCP/IP}
\section{Distibuted Application Facility}
The smallest part of RINA is the Distributed Application Process (DAP) which is a process running on a host,
if at least two of them communicate it is called a Distributed Application Facility (DAF). They communicate 
with objects structured in a Resource Information Base (RIB) that define naming and structure. They use
a Common Distributed Application Protocol (CDAP) that permit to execute 6 operations on a distant DAP's
objects:
\begin{itemize}
\item create
\item delete
\item read
\item start
\item stop
\item write
\end{itemize}
The DAF can be compared to the Application layer on the TCP/IP model. For multiple DAPs to communicate, they 
need underlying `layers'.\cite{wiki}

\part{Security induced by RINA's conception}

\part{(optionnel) The threats to RINA}

\nocite{*}
\newpage
\bibliographystyle{unsrt}
\bibliography{report}
\end{document}
