\documentclass[a4paper]{proc}

\usepackage[utf8]{inputenc}
\usepackage[T1]{fontenc}
\usepackage[english]{babel}
\usepackage{graphicx}
\usepackage{url}
\usepackage{indentfirst}

\author{Paul MABILEAU\\\texttt{paul.mabileau@telecom-sudparis.eu} \and Franck STAUFFER\\\texttt{franck.stauffer@telecom-sudparis.eu}}
\title{\textbf{Recursive Inter-Network Architecture's Security}}

\begin{document}
\maketitle
\tableofcontents
\newpage
\part*{Introduction}
The Recursive Inter-Network Architecture (RINA) is a network architecture based on Internet Process Communication (IPC) presented as an alternative to TCP/IP\@.
It was designed a while ago without security in mind. 
It has been patched to include security but it creates complexity.\cite{assessing-security}

\part{Comparing RINA against TCP/IP}
\section{Distibuted Application Facility}
The smallest part of RINA is the Distributed Application Process (DAP) which is a process running on a host, if at least two of them communicate it is called a Distributed Application Facility (DAF).
They communicate with objects structured in a Resource Information Base (RIB) that define naming and structure. (figure~\ref{daf})\\
They use a Common Distributed Application Protocol (CDAP) that permit to execute 6 operations on a distant DAP's objects:
\begin{itemize}
\item create
\item delete
\item read
\item start
\item stop
\item write
\end{itemize}
The DAF can be compared to the Application layer on the TCP/IP model. DAPs need underlying `layers' to communicate.\cite{wiki}

\begin{figure}
    \centering
    \includegraphics[width=0.9\columnwidth]{DAF.png}\label{daf}\caption{Base of RINA's architecture}
\end{figure}

\section{Distributed IPC Facility}
A Distributed IPC Facility (DIF) is a DAF but instead of containing DAPs, it contains IPC Processes (IPCPs).
It can be compared to a layer in the TCP/IP model but there is not a fixed amount of DIFs, the number can be adapted to the needs of a specific network.

\part{Security provided by RINA's conception}
\section{Authentication}

\par In RINA, authentication is mandatory. This is one of the major, if not the
most important security feature of the architecture, as previous security
analysis often consider attacks plausible only after a successful authentication
\cite{assessing-security, wiki, PINS}.

\par Authentication occurs before a process may communicate on a DIF, either by
creating a new one, or by joining an existing one, and is verified by the
destination application process. When creating a brand new DIF, the initial IPC
process simply has to exist in order to let others join the newly created DIF.
When joining an already existing DIF, the joining process asks permission to the
destination process it wants to communicate with. It achieves this using a
subsequent DIF that both processes have in common and are authenticated in.
The destination process is the one that determines whether the joining one has
sufficient rights through any sort of authentication, described in the DIF's
policy. It may be as strong or weak as desired by the DIF and for the wanted
communication. Then, the destination process gives the joining one a new
application name so it may communicate on the newly joined DIF with.

\par This design feature is quite useful for hardening security considerations
in network communications, as being authenticated \textit{every time} means that
the destination application decides who to talk with and how strict this
decision is using policies, \textit{every time}. It thus builds greater trust
between applications through native access control.

\part{(optional) The threats to RINA}

\nocite{*}
\newpage
\bibliographystyle{unsrt}
\bibliography{report}
\end{document}
